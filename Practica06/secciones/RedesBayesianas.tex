% ----------------------------------------------------------------------------------------\
% ---------------------------------------------------------------------------------------\
% --------------------------------------------------------------------------------------\
\section{Investigación}
% ----------------------------------------------------------------------------------------\
% ---------------------------------------------------------------------------------------\
% --------------------------------------------------------------------------------------\

% ----------------------------------------------------------------------------------------\
% ---------------------------------------------------------------------------------------\
\subsection{Fundamentos de redes bayesianas}
% ----------------------------------------------------------------------------------------\
% ---------------------------------------------------------------------------------------\


% ----------------------------------------------------------------------------------------\
% Investiga y redacta un breve resumen (máximo una cuartilla) sobre
%  qué son las redes bayesianas y para qué se utilizan.
\subsection*{Redes bayesianas}
% ----------------------------------------------------------------------------------------\
Una red Bayesiana se define como un par \( (G, \Theta) \), donde \( G \) es un grafo dirigido acíclico (DAG) y \( \Theta \) es un conjunto de tablas de probabilidad condicional (TPC). 

El grafo \( G \) consiste en un conjunto de nodos \( V \) y un conjunto de arcos \( E \). Cada nodo representa una variable aleatoria y cada arco representa una dependencia probabilística entre las variables. Las tablas de probabilidad condicional \( \Theta \) especifican la probabilidad de cada valor posible de cada variable condicionado a los valores de sus nodos padres.

Para representar la probabilidad conjunta de todas las variables en la red, utilizamos la regla del producto:

\[
P(\text{{variables}}) = \prod_{i=1}^{n} P(X_i \,|\, \text{{padres}}(X_i))
\]

donde \( X_i \) es la variable \( i \)-ésima en la red y \( \text{{padres}}(X_i) \) son los nodos padres de \( X_i \) en el grafo.

Por ejemplo, supongamos que tenemos dos variables \( A \) y \( B \) en una red Bayesiana, donde \( B \) depende de \( A \). Podemos representar esto como:

\[
P(A, B) = P(A) \cdot P(B \,|\, A)
\]

Aquí, \( P(A) \) es la probabilidad marginal de \( A \) y \( P(B \,|\, A) \) es la probabilidad condicional de \( B \) dado \( A \). La probabilidad conjunta de \( A \) y \( B \) se calcula multiplicando la probabilidad de \( A \) por la probabilidad condicional de \( B \) dado \( A \).

Las redes Bayesianas se utilizan en una amplia gama de aplicaciones. Por ejemplo:

\begin{itemize}
    \item  En medicina, Las redes Bayesianas pueden ayudar a los médicos a evaluar la probabilidad de que un paciente tenga cierta enfermedad en función de los síntomas observados y los factores de riesgo.
    \item En finanzas, pueden utilizarse para evaluar riesgos y rendimientos en carteras de inversión.
    \item En reconocimiento de patrones, pueden utilizarse para clasificar imágenes o datos sensoriales.

    \item Predicción del tiempo: Pueden utilizarse para modelar la probabilidad de diferentes condiciones climáticas en función de variables como la temperatura, la presión atmosférica y la humedad.

\end{itemize}
En cada una de estas aplicaciones, las redes Bayesianas permiten modelar la incertidumbre y realizar inferencias probabilísticas sobre las variables de interés.

% ----------------------------------------------------------------------------------------\
% Explica la diferencia entre probabilidad condicional e independencia condicional, 
%  incliyendo ejemplos claros de cada uno.
\subsection*{Probabilidad condicional e independencia condicional}
% ----------------------------------------------------------------------------------------\

\begin{enumerate}
    \item Probabilidad Condicional:

La probabilidad condicional se refiere a la probabilidad de que ocurra un evento dado que otro evento ya ha ocurrido. Se denota como \( P(A | B) \), que es la probabilidad de que el evento \( A \) ocurra dado que el evento \( B \) ha ocurrido. La fórmula para calcular la probabilidad condicional es:

\[ P(A | B) = \frac{P(A \cap B)}{P(B)} \]

Lo que significa que la probabilidad de \( A \) dado \( B \) es igual a la probabilidad de que ocurran ambos \( A \) y \( B \) dividida por la probabilidad de que ocurra \( B \).

Un ejemplo es: supongamos que tenemos dos eventos \( A \) y \( B \), donde \( A \) representa sacar una carta roja de una baraja estándar de 52 cartas, y \( B \) representa sacar una carta de corazones. La probabilidad de sacar una carta roja dado que hemos sacado una carta de corazones se puede calcular utilizando la fórmula de probabilidad condicional:

\[ P(A | B) = \frac{P(A \cap B)}{P(B)} \]

Donde \( P(A \cap B) \) es la probabilidad de sacar una carta roja que también sea un corazón, y \( P(B) \) es la probabilidad de sacar una carta de corazones. Si la carta roja que hemos sacado también es un corazón, entonces la probabilidad de \( A \) dado \( B \) sería \( 1/13 \), ya que hay 13 cartas de corazones en total en una baraja de 52 cartas.

\item Independencia Condicional:

Es la situación en la que dos eventos son independientes dado un tercer evento. Formalmente, dos eventos \( A \) y \( B \) son independientes condicionalmente a un evento \( C \) si la probabilidad de \( A \) dado \( B \) no cambia incluso si sabemos que \( C \) ha ocurrido. Se denota como \( A \perp\!\!\!\perp B \,|\, C \).

Por ejemplo, consideremos tres eventos \( A \), \( B \) y \( C \), donde \( A \) representa sacar una carta roja de una baraja de cartas, \( B \) representa sacar una carta de corazones y \( C \) representa sacar una carta alta (as, rey, reina o jota). Si los eventos \( A \) y \( B \) son independientes condicionalmente a \( C \), esto significa que la probabilidad de sacar una carta roja que también sea un corazón no cambia incluso si sabemos que hemos sacado una carta alta. En este caso, si \( A \) y \( B \) son independientes condicionalmente a \( C \), entonces la probabilidad de \( A \) dado \( B \) sigue siendo \( 1/13 \), incluso después de que hemos sacado una carta alta.

% ----------------------------------------------------------------------------------------\
% Describe cómo se representa matemáticamente una red bayesiana y
%  el significado de los nodos y aristas en cada representación.
\subsection*{Representación Matematica}
% ----------------------------------------------------------------------------------------\

