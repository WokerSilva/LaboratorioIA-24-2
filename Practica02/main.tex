\documentclass[a4paper,12pt]{article} 
\usepackage[utf8]{inputenc} % Acentos válidos sin problemas
\usepackage[spanish]{babel} % Idioma

\usepackage[backend=biber, style=alphabetic, sorting=ynt]{biblatex}
\bibliography{bibliografia.bib}
\nocite{*} % Añade todas las referencias de bib sin cita

\input{packet}
\pagestyle{fancy}

\begin{document}%----------------------INICIO DOCUMENTO------------|
%------------------------------------------------------------------|
\begin{titlepage}
\center 
\newcommand{\HRule}{\rule{\linewidth}{0.5mm}} 

%---------------------
%	ESCUDO
%---------------------
\includegraphics[width=4.5cm]{IMA/cienciasWhite.png}

%----------------------------
%	TITULO
%----------------------------
\quad \\[0.2cm]
\textsc{\huge Facultad de Ciencias}\\[.6cm] 
\textsc{\huge Inteligencia Artificial}\\[0.5cm]

%-------------
%	TRABAJO
%-------------
\makeatletter
    \HRule \\ [0.4cm]
        { \huge \bfseries Exploradores de laberinto}\\
    \HRule \\ [0.4cm]
    
\vspace{2mm}

%----------------------------
%	Nombre del Equipo
%----------------------------
\begin{flushleft}
    \Large{Equipo:} \texttt{\Large Skynet Scribes}
\end{flushleft}
%----------------------------
%	Número de practica
%----------------------------
\begin{flushleft}
    \Large{Número de practica:} \texttt{\Large 02}\\[0.8cm]
\end{flushleft}


%-------------------
%	AUTORES
%-------------------
\begin{minipage}{0.8\textwidth}
    \begin{flushright}
        \textbf{\large{Carlos Daniel Cortés Jiménez}}\\    
        420004846        
    \end{flushright}
\end{minipage}

\vspace{5mm}

\begin{minipage}{0.4\textwidth}
        \textbf{\large{Sarah Sophía Olivares García}}\\
        318360638
\end{minipage}
\begin{minipage}{0.4\textwidth}
    \begin{flushright}
        \textbf{\large{Marco Silva Huerta}}\\
        316205326        
    \end{flushright}
\end{minipage}

\vspace{5mm}

\begin{minipage}{0.4\textwidth}
        \textbf{\large{Juan Daniel Barrera Holan}}\\    
        417079372
\end{minipage}
\begin{minipage}{0.4\textwidth}
    \begin{flushright}
        \textbf{\large{Laura Itzel Tinoco Miguel}}\\
        316020189
    \end{flushright}
\end{minipage}

\vspace{10mm}
%-------------------
%	PROFESORES
%-------------------

\begin{minipage}{0.8\textwidth}
    \begin{flushleft} \large
        Profesora: Cecilia Reyes Peña\\
        Ayudante teoría: Karem Ramos Calpulalpan \\
        Ayudante laboratorio: Tania Michelle Rubí Rojas\\                    
    \end{flushleft}
\end{minipage}

\vspace{20mm}

\begin{minipage}{0.4\textwidth}
    %---------------
    %	S E M E S T R E
    %---------------
    \textcolor{white}{Semestre}\\
    \large\textbf{Semestre 2024-2}      
\end{minipage}
\begin{minipage}{0.4\textwidth}
    %---------------
    %	F E C H A
    %---------------
    \begin{flushright}
        {\large Fecha de entrega:\\
         \textbf{28 de Febrero del 2024}}
    \end{flushright}
\end{minipage}

\makeatother

\vfill 
\end{titlepage}

\newpage
% ------------------------------------------------------------------------------|
% -----------------------------------------------------------------------------|
% ----------------------------------------------------------------------------|
\section{Backtracking}
% Media Cuartilla de investigación sobre origen, uso y aplicaciones
% ----------------------------------------------------------------------------\
% -----------------------------------------------------------------------------\
% ------------------------------------------------------------------------------\





% ------------------------------------------------------------------------------|
% -----------------------------------------------------------------------------|
% ----------------------------------------------------------------------------|
\section{Algoritmos similares}
% Media Cuartilla de investigación sobre la relación con otras soluciones
Algunos algoritmos que utilizan estrategias similares a Backtracking son:

\begin{itemize}
    \item \textbf{Branch and Bound:} Este realiza un recorrido sistemático del árbol de estados de un problema, si bien ese recorrido no tiene por qué ser en profundidad, como sucedía en backtracking: usaremos una estrategia de ramificación. Aqui se utilizan tecnicas de poda para poder eliminar aquellos nodos que no lleven a solucines óptimas.
    
    \item \textbf{Depth-First Search (DFS):} Este es un algoritmo de búsqueda no informada, lo que hace primero es proporcionar los pasos para atravesar todos los nodos de un gráfico sin repetir ningún nodo.
    
    \item \textbf{Breadth-First Search (BFS):}Es un algoritmo para atravesar o buscar estructuras de datos de árboles. Comienza en la raíz del árbol (o algún nodo arbitrario, en ocasiones denominado "clave de búsqueda") y explora primero los nodos vecinos antes de pasar a los vecinos del siguiente nivel.
    
    \item \textbf{A (A estrella):*} A* combina la búsqueda heurística con la búsqueda informada para encontrar la solución más eficiente en problemas de ruta. Aunque difiere del backtracking, comparten el objetivo de encontrar soluciones en un espacio de búsqueda.
\end{itemize}





% ------------------------------------------------------------------------------|
% -----------------------------------------------------------------------------|
% ----------------------------------------------------------------------------|
\section{Pseudocódigo}
% Agregar un pseudocódigo sobre el algoritmo
% ----------------------------------------------------------------------------\
% -----------------------------------------------------------------------------\
% ------------------------------------------------------------------------------\

\begin{verbatim}
    # aquí va el código 
\end{verbatim}









% ------------------------------------------------------------------------------|
% -----------------------------------------------------------------------------|
% ----------------------------------------------------------------------------|
\section{Documentación}
% ----------------------------------------------------------------------------\
% -----------------------------------------------------------------------------\
% ------------------------------------------------------------------------------\





% ----------------------|
% Referencias           |
% Forma de Compilar     |
% pdflatex main.tex     |
% biber main            |
% pdflatex main.tex     |
\newpage %              |
\thispagestyle{fancyref}
\printbibliography %    |
% ----------------------|

\end{document}%----------------------F I N DOCUMENTO---------------|
%------------------------------------------------------------------|
