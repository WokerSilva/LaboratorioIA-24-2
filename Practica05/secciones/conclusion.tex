% ----------------------------------------------------------------------------------------\
% ---------------------------------------------------------------------------------------\
% --------------------------------------------------------------------------------------\
\section{Resultados obtenidos}
% ----------------------------------------------------------------------------------------\
% ---------------------------------------------------------------------------------------\
% --------------------------------------------------------------------------------------\
% Mostrar ejecuciones del código con capturas 
% Mostrar una comparación del código

% ----------------------------------------------------------------------------------------\
% ---------------------------------------------------------------------------------------\
\subsection{Implementación básica del Juego de la Vida}
% ----------------------------------------------------------------------------------------\
% ---------------------------------------------------------------------------------------\




% ----------------------------------------------------------------------------------------\
% ---------------------------------------------------------------------------------------\
\subsection{Implementación de Algoritmos Genéticos en el Juego de la Vida}
% ----------------------------------------------------------------------------------------\
% ---------------------------------------------------------------------------------------\

\subsubsection*{Animación}

Usamos la importación de \texttt{import matplotlib} para usar \texttt{matplotlib.use('TkAgg')}
y así poder ejecutar la ventana emergente que muestra la animación del Juego de la vida. 
Mencionamos esto por dos cosas:

\begin{itemize}
    \item Notamos que al ejecutar un archivo \texttt{.py} funciona, al ejecutar un archivo 
    \textit{.ipynb} en nuestro caso dentro de Visual Studio Code funciona la animación pero si
    el código es ejecutado en un archivo de \textbf{colab} truena y aunque quitemos la 
    importación la animación no se ejecuta. 

    \item Se necesito instalar usando el comando (para linux): 
    \texttt{sudo apt install python3-tk}, es para la visualización de gráficos.
\end{itemize}

\subsubsection*{Pruebas}

Nuestro primer caso de salida es cuando hemos agotado la población y no hay 
suficientes para realizar el torneo.
\begin{center}
    \includegraphics[scale = .4]{IMA/selecTorneo.png}
\end{center}

La segunda salida es cuando la generación no es apta bajo los criterios, por 
lo tanto no hay una nueva generación y tenemos población vacía, entonces 
terminamos.
\begin{center}
    \includegraphics[scale = .4]{IMA/poblacionVacia.png}
\end{center}

Para la tercera salida es el caso especial que condicionamos, donde si nuestro 
porcentaje de células vivas no es mayor después de cierta generación (elegimos
el 70 porciento del total dado) salimos del programa. Esto lo hicimos con la 
finalidad de encontrar células más aptas en menos generaciones. 
\begin{center}
    \includegraphics[scale = .4]{IMA/condicionEspecial.png}
\end{center}

Finalmente el caso exitoso donde llegamos al final de las generaciones establecidas. 
\begin{center}
    \includegraphics[scale = .4]{IMA/casoExitoso.png}
\end{center}

\subsubsection*{Notas}


Dentro del algoritmo se definieron constantes con las que hemos jugado a lo largo de 
la creación del algoritmo, comprobamos la importancia de la función de aptitud para
llegar a nuestro objetivo a demás de ver como las mutaciones en este caso particular 
pueden ser buenas y malas. 

Las constantes son:
\begin{itemize}
    \item \texttt{TABLERO} Se define un tablero cuadrado y podemos jugar con el tamaño
    de sus lados para que a medida que células más aptas crezcan podamos observar patrones
    generados, cosa que no es tan visual con tableos pequeños.

    \item \texttt{POBLACION INICIAL} Comprobamos que para nuestra solución entre menos 
    población habrá menos oportunidades de éxito, una posible mejora para tratar eso sería 
    modificar nuestra función de torneo. 

    \item \texttt{GENERACIONES} Son la medida para saber que tanto funciona nuestra función 
    de aptitud junto con el número de células vivas objetivo. 

    \item \texttt{OBJETIVO CELULAS VIVAS} El porcentaje que deseamos este cubierto de células 
    vivas dentro del tablero.

    \item \texttt{C} Es la constante que nos ayuda para la formula de las astronaves.   
\end{itemize}

% ----------------------------------------------------------------------------------------\
% ---------------------------------------------------------------------------------------\
% --------------------------------------------------------------------------------------\
\section{Reflexión final}
% ----------------------------------------------------------------------------------------\
% ---------------------------------------------------------------------------------------\
% --------------------------------------------------------------------------------------\

% Después de las simulaciones, analizar cómo la evolución de los cromosomas afectó el 
% desarrollo del Juego de la Vida, identificando patrones o estrategias exitosas.

% Redactar un breve informe que reflexione sobre el aprendizaje obtenido, las estrategias 
% que resultaron ser más efectivas y cómo los principios de los algoritmos genéticos 
% podrían aplicarse a otros problemas de optimización o simulación.