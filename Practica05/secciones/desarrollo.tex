% ----------------------------------------------------------------------------------------\
% ---------------------------------------------------------------------------------------\
% --------------------------------------------------------------------------------------\
\section{Desarrollo}
% ----------------------------------------------------------------------------------------\
% ---------------------------------------------------------------------------------------\
% --------------------------------------------------------------------------------------\
% Explicación de las implementaciones, diagrama de flujo, ideas, comentarios, investigación, etc
% ----------------------------------------------------------------------------------------\
% ---------------------------------------------------------------------------------------\
\subsection{Implementación básica del Juego de la Vida}
% ----------------------------------------------------------------------------------------\
% ---------------------------------------------------------------------------------------\



% ----------------------------------------------------------------------------------------\
% ---------------------------------------------------------------------------------------\
\subsection{Introducción a los Algoritmos Genéticos}
% ----------------------------------------------------------------------------------------\
% ---------------------------------------------------------------------------------------\



% ----------------------------------------------------------------------------------------\
% ---------------------------------------------------------------------------------------\
\subsection{Implementación de Algoritmos Genéticos en el Juego de la Vida}
% ----------------------------------------------------------------------------------------\
% ---------------------------------------------------------------------------------------\

Recordemos las reglas iniciales del juego:
\begin{enumerate}
    \item Si una célula está viva y tiene dos o tres vecinas vivas, sobrevive.
    \item Si una célula está muerta y tiene tres vecinas vivas, nace.
    \item Si una célula está viva y tiene más de tres vecinas vivas, muere.
    
    La disposición o patrón inicial de células se llama \textit{semilla}. La siguiente 
    generación nace de aplicar las reglas del juego a todas las células de manera 
    simultánea. Este proceso se puede ejecutar de manera indefinida.
\end{enumerate}

\subsubsection*{Modificación de las Reglas}

Para modificar la simulación del Juego por los cromosomas de una población podemos crear
una \textbf{representación cromosómica:} donde cada cromosoma será una cadena binaria 
donde cada bit representa el estado de una célula en el tablero (vivo o muerto).\\ 

Así las nuevas reglas para el \textit{Juego de la Vida} que estamos proponiendo es para 
buscar crear más vida con menos generaciones, las células  podrán vivir con más vecinos 
si se alcanza un cierto número de cromosomas en la población:

\begin{itemize}
    \item Nacimientos: Una célula muerta con exactamente tres vecinos vivos se convierte 
    en una célula viva.
    \item Muerte uno: Una célula viva con uno o menos vecinos vivos muere.
    \item Muerte dos: Una célula viva con más de tres vecinos vivos muere, a menos que se 
    cumpla la condición especial.
    \item Condición Especial de Supervivencia: Si la población de cromosomas alcanza o 
    supera un $n$ especifica (80), las células vivas pueden soportar hasta cuatro 
    vecinos vivos sin morir.
    \item Supervivencia Normal: Si no se cumple la condición especial, una célula viva con 
    dos o tres vecinos vivos sobrevive.
    \item Muerte tres: Si en $n$ generaciones no se alcanza el objetivo el juego termina. (80)
\end{itemize}

Este cambio significa querer a los mejores individuos que logren alcanzar el objetivo creando 
nuestra función \textbf{fitness} con mucho cuidado ya que por medio de esta  haremos que 
nuestra población no muera.\\ 


\subsubsection*{Pasos del algoritmo}


\begin{itemize}
    \item Inicialización de la Población: Generamos una población inicial de cromosomas de manera aleatoria
    \item Evaluación de la Aptitud: DEFINIR
    \item Selección: Utilizamos un método de selección por torneo
    \item Reproducción: Cruzamiento en dos puntos, en el que se intercambian los genes que aparecen en el intervalo de genes delimitados por dos puntos.
    \item Mutación: Cambiar aleatoriamente el estado de algunas células
    \item Remplazo: Una vez realizada la mutación y aplicar una evaluación fitness (más tranquila que la evaluación de la aptitud) se hace el remplazo
\end{itemize}