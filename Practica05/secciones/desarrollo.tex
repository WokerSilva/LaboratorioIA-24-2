% ----------------------------------------------------------------------------------------\
% ---------------------------------------------------------------------------------------\
% --------------------------------------------------------------------------------------\
\section{Desarrollo}
% ----------------------------------------------------------------------------------------\
% ---------------------------------------------------------------------------------------\
% --------------------------------------------------------------------------------------\
% Explicación de las implementaciones, diagrama de flujo, ideas, comentarios, investigación, etc
% ----------------------------------------------------------------------------------------\
% ---------------------------------------------------------------------------------------\
\subsection{Implementación básica del Juego de la Vida}
% ----------------------------------------------------------------------------------------\
% ---------------------------------------------------------------------------------------\
    
\subsubsection*{Descripción del código}

El código implementa el Juego de la Vida, un autómata celular que simula el comportamiento de 
un sistema de células vivas y muertas en una cuadrícula, el código consta de las siguientes 
partes principales:

\subsubsection*{Creación del estado inicial del juego}

La función \textbf{crear\_matriz\_inicial} genera una matriz cuadrada de tamaño $n \times n$ con 
valores aleatorios de 0 (célula muerta) y 1 (célula viva), esta matriz representa el estado 
inicial del juego.

\begin{lstlisting}
def crear_matriz_inicial(n):
    """
    Genera una matriz n x n con celulas vivas o muertas aleatoriamente.
    
    Parametros:
    - n: Tamano de la matriz (numero de filas y columnas).
    
    Retorna:
    - Matriz n x n con valores aleatorios 0 (muerta) o 1 (viva).
    """
    return np.random.choice([0, 1], size=(n, n))
\end{lstlisting}


\subsubsection*{Aplicación de las reglas del Juego de la Vida}

La función \textbf{aplicar\_reglas} toma el estado actual del juego como entrada y aplica las 
reglas del Juego de la Vida para determinar el siguiente estado, las reglas son:

\begin{itemize}
    \item Una célula viva con dos o tres vecinas vivas sobrevive.
    \item Una célula muerta con tres vecinas vivas nace.
    \item En cualquier otro caso, la célula muere.
\end{itemize}

\begin{lstlisting}
def aplicar_reglas(estado_actual):
    """
    Aplica las reglas del Juego de la Vida a cada celula de la matriz.
    
    Parametros:
    - estado_actual: Matriz que representa el estado actual del juego.
    
    Retorna:
    - Nuevo estado del juego despues de aplicar las reglas.
    """
    n = estado_actual.shape[0]
    nuevo_estado = np.zeros((n, n))
    
    for i in range(n):
        for j in range(n):
            num_vecinos = np.sum(estado_actual[i-1:i+2, j-1:j+2]) - estado_actual[i, j]
            
            # Regla a) Si una celula esta viva y tiene dos o tres vecinas vivas, sobrevive.
            if estado_actual[i, j] == 1 and num_vecinos in [2, 3]:
                nuevo_estado[i, j] = 1
                        
            # Regla b) Si una celula esta muerta y tiene tres vecinas vivas, nace.
            elif estado_actual[i, j] == 0 and num_vecinos == 3:
                nuevo_estado[i, j] = 1
            else:
                nuevo_estado[i, j] = 0
    
    return nuevo_estado
\end{lstlisting}

\subsubsection*{Visualización del estado del juego}

La función \textbf{visualizar\_estado} se utiliza para mostrar el estado actual del 
juego mediante una imagen en escala de grises.

\begin{lstlisting}
def visualizar_estado(estado, titulo="Estado del Juego de la Vida"):
    """
    Visualiza el estado del juego.
    
    Parametros:
    - estado: Matriz que representa el estado actual del juego.
    - titulo: Titulo para la visualizacion.
    """
    plt.figure(figsize=(5, 5))
    plt.imshow(estado, cmap='Greys', interpolation='nearest')
    plt.title(titulo)
    plt.show()
\end{lstlisting}

\subsubsection*{Actualización de la matriz y visualización a través de múltiples turnos}

Esta sección del código implementa un bucle que actualiza la matriz basándose en las reglas y 
visualiza el estado de la cuadrícula después de cada actualización.

\begin{lstlisting}
# Numero de iteraciones/turnos para visualizar
num_iteraciones = 5
n = 10  # Tamano de la matriz

# Generar el estado inicial del juego
estado_actual = crear_matriz_inicial(n)

# Visualizar el estado inicial
visualizar_estado(estado_actual, "Estado Inicial")

# Bucle para actualizar y visualizar el juego en cada turno
for i in range(num_iteraciones):
    # Aplicar las reglas del Juego de la Vida para obtener el nuevo estado
    estado_actual = aplicar_reglas(estado_actual)
    # Visualizar el estado actual despues de aplicar las reglas
    visualizar_estado(estado_actual, titulo=f"Estado despues de {i + 1} turno(s)")
\end{lstlisting}


% ----------------------------------------------------------------------------------------\
% ---------------------------------------------------------------------------------------\
\subsection{Introducción a los Algoritmos Genéticos}
% ----------------------------------------------------------------------------------------\
% ---------------------------------------------------------------------------------------\






% ----------------------------------------------------------------------------------------\
% ---------------------------------------------------------------------------------------\
\subsection{Implementación de Algoritmos Genéticos en el Juego de la Vida}
% ----------------------------------------------------------------------------------------\
% ---------------------------------------------------------------------------------------\

Recordemos las reglas iniciales del juego:
\begin{enumerate}
    \item Si una célula está viva y tiene dos o tres vecinas vivas, sobrevive.
    \item Si una célula está muerta y tiene tres vecinas vivas, nace.
    \item Si una célula está viva y tiene más de tres vecinas vivas, muere.
    
    La disposición o patrón inicial de células se llama \textit{semilla}. La siguiente 
    generación nace de aplicar las reglas del juego a todas las células de manera 
    simultánea. Este proceso se puede ejecutar de manera indefinida.
\end{enumerate}

% ----------------------------------------------------------------------------------------\
\subsubsection*{Modificación de las Reglas}
% ----------------------------------------------------------------------------------------\

Para modificar la simulación del Juego por los cromosomas de una población podemos crear
una \textbf{representación cromosómica:} donde cada cromosoma será una cadena binaria 
donde cada bit representa el estado de una célula en el tablero (vivo o muerto).\\ 

Así las nuevas reglas para el \textit{Juego de la Vida} que estamos proponiendo es para 
buscar crear más vida con menos generaciones, las células  podrán vivir con más vecinos 
si se alcanza un cierto número de cromosomas en la población:

\begin{itemize}
    \item Nacimientos: Una célula muerta con exactamente tres vecinos vivos se convierte 
    en una célula viva.
    \item Muerte uno: Una célula viva con uno o menos vecinos vivos muere.
    \item Muerte dos: Una célula viva con más de tres vecinos vivos muere, a menos que se 
    cumpla la condición especial.
    \item Condición Especial de Supervivencia: Si la población de cromosomas alcanza o 
    supera un $n$ especifica, las células vivas pueden soportar hasta cuatro 
    vecinos vivos sin morir.
    \item Supervivencia Normal: Si no se cumple la condición especial, una célula viva con 
    dos o tres vecinos vivos sobrevive.
    \item Muerte tres: Si en $n$ generaciones no se alcanza el objetivo el juego termina.
\end{itemize}

El objetivo que queremos lograr con este cambio es encontrar un conjunto de células más eficiente
en el menor tiempo de generaciones, para este caso definiremos que sea cumplido tras un 
70 porciento del total de generaciones (aproximadamente 160 generaciones), es decir, buscamos 
las células que se mantengan como población estable y amplia cumpliendo los criterios de aptitud.

% ----------------------------------------------------------------------------------------\
\subsubsection*{Pasos del algoritmo}
% ----------------------------------------------------------------------------------------\

\begin{itemize}
    \item Inicialización de la Población: Generamos una población inicial de cromosomas de 
    manera aleatoria
    \item Evaluación de la Aptitud: astronaves
    \item Selección: Utilizamos un método de selección por torneo
    \item Reproducción: Cruzamiento en dos puntos, en el que se intercambian los genes que 
    aparecen en el intervalo de genes delimitados por dos puntos.
    \item Mutación: Cambiar aleatoriamente el estado de algunas células
    \item Remplazo: Una vez realizada la mutación y aplicar una evaluación fitness (más 
    tranquila que la evaluación de la aptitud) se hace el remplazo
\end{itemize}


% ----------------------------------------------------------------------------------------\
\subsubsection*{Inicialización de la Población}
% ----------------------------------------------------------------------------------------\

Cada célula en el tablero de $n \times n$ tiene ocho vecinos, que incluyen las celdas 
adyacentes horizontal, vertical y diagonalmente. Al comienzo del juego, la población inicial 
se genera aleatoriamente, cada cromosoma es una secuencia binaria que representa un estado 
completo del tablero, con 1s para las células vivas y 0s para las muertas.

\subsubsection*{Función fitness}

Para definir la función fitness necesitamos saber que existen patrones básicos dentro del 
juego y estos son configuraciones de los vecinos para las células que determinan un 
comportamiento concreto como patrones estáticos que no hay nacimientos ni fallecimientos,
y nunca cambian, patrones recurrentes o \textit{osciladores} que evolucionan a través de 
diversos estados pero vuelven a su forma inicial y patrones que se trasladan por el tablero
llamados \textit{spaceships}.\\ 

Las astronaves \cite{spaceships} son patrones que se caracterizan por desplazarse a través 
del tablero a lo largo del tiempo, bien sea de forma diagonal o de forma horizontal o 
vertical. La velocidad de desplazamiento es variable, dependiendo del patrón que se trate.
y esta capacidad para moverse eficientemente es un indicador de su robustez y estabilidad 
dentro del juego\\ 

\noindent La fórmula de velocidad para las \textit{spaceships}:

\begin{equation*}
    v = \frac{max(|x|,|y|)}{n} \times c
\end{equation*}

\begin{itemize}
    \item[] ($v$) es la velocidad de la astronave.
    \item[] ($x$) y ($y$) son los desplazamientos en dos dimensiones (horizontal y vertical).
    \item[] ($n$) es el número de generaciones que tarda la astronave en desplazarse.
    \item[] ($c$) es una constante análoga a la velocidad de la luz, que en este contexto se 
    puede considerar como 1 celda por generación.
\end{itemize}

Esta fórmula es útil para medir la eficiencia con la que un patrón se desplaza a través del 
tablero. Los patrones que se mueven más rápido (mayor valor de ($v$)) serían considerados 
más \textit{aptos} mientras que con velocidades más bajas nos dice que esos patrones son menos
eficientes (lentos) o que mueren rápidamente, de manera que calificaran con aptitud menor.\\    


Entonces estamos considerando la distancia máxima recorrida en cualquier dirección y el 
número de generaciones necesarias para dicho desplazamiento, la ventaja de usar la velocidad
para hacer la evaluación es que favorece patrones que pueden sobrevivir y moverse a lo largo de 
las generaciones, como si fuera una evolución natural sin embargo se añadió otro criterio el 
cual consiste en sumar puntos si es que los patrones no chocan demasiado con los bordes, ya que 
células vivas en bordes podrían tener menos vecinos para sobrevivir. 