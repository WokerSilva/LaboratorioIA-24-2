\section{Resultados obtenidos}

\subsection*{Ejemplos usados}

\subsection*{Observaciones}


\subsection*{Versiones del codigo}

En seguida haremos notar los cambios que se hicieron entre dos versiones de implementacion y el por que de estos cambios.

\begin{lstlisting}[language=Python, caption=Version 1: reconstruir_camino]
    def reconstruir_camino(casilla_actual):
    """
    Esta función reconstruye el camino desde la casilla actual hasta la posición inicial del agente.

    recibe:
        casilla_actual (Agente): La casilla actual.

    devuelve:
        list: Una lista que representa el camino desde la casilla actual hasta la posición inicial del agente.
    """
    camino = []
    while casilla_actual:
        camino.append(casilla_actual.posicion)
        casilla_actual = casilla_actual.padre

    return list(reversed(camino))
\end{lstlisting}

\begin{lstlisting}[language=Python, caption=Version 2: reconstruir_camino]
    def reconstruir_camino(casilla_actual):
    """
    Esta función reconstruye el camino desde la casilla actual hasta la posición inicial del agente.

    recibe:
        casilla_actual (Agente): La casilla actual.

    devuelve:
        list: Una lista que representa el camino desde la casilla actual hasta la posición inicial del agente.
    """
    camino = []
    while casilla_actual:
        camino.append(casilla_actual.posicion)
        casilla_actual = casilla_actual.padre

    camino = list(reversed(camino))
    print("Ruta seguida:")
    for posicion in camino:
        print(f"[{posicion[0]}, {posicion[1]}]")
    print("Se encontró la salida :D")  

    print("\nCamino encontrado:")
    for i in range(len(laberinto)):
        fila = ""
        for j in range(len(laberinto[0])):
            if [i, j] in camino:
                fila += "X "
            else:
                fila += "- "
        print(fila)

    return camino
\end{lstlisting}

{\tt Notemos que en la version 2 el cambio es mas que nada la salida por pantalla, es decir la ruta que el agente tomo para encontrar la salida, y ademas mosyramos el camino seguido por el agente, aquí iteramos sobre el laberinto y determinamos la posición del agente, si se encuentra dicha posición colocamos una \textbf(X), en caso contrario solo coloacamos una guion \textbf(-). Este cambio lo consideramos importante ya que como bien se dijo anteriormente hace que se mas visible, clara y permite que se pueda entender mejor como se llego a la salida.}

\begin{lstlisting}[language=Python, caption=Version 1: distancia_manhattan]
    def distancia_manhattan(nodo_actual, nodo_objetivo):
    """
    Calcula la distancia de Manhattan entre dos nodos.

    recibe:
        nodo_actual (Agente): El nodo actual.
        nodo_objetivo (Agente): El nodo objetivo.

    devuelve:
        int: La distancia de Manhattan entre los dos nodos.
    """
    diferencia_x = nodo_actual.posicion[0] - nodo_objetivo.posicion[0]
    diferencia_y = nodo_actual.posicion[1] - nodo_objetivo.posicion[1]
    distancia = abs(diferencia_x) + abs(diferencia_y)
    return distancia
\end{lstlisting}

\begin{lstlisting}[language=Python, caption=Version 2: distancia_manhattan]
    def distancia_manhattan(laberinto):
    """
    Calcula la distancia de Manhattan entre dos nodos.

    recibe:
        laberinto (list): El laberinto representado como una matriz

    devuelve:
        int: La distancia de Manhattan entre los dos nodos.
    """
    nodo_actual = punto_de_partida(laberinto) #Buscamos la posicion inicial del agente
    nodo_objetivo = salida_laberinto(laberinto) #Buscamos la salida del laberinto

    x, y = nodo_actual 
    w, z = nodo_objetivo
    
    distancia = abs(x - w) + abs(y - z) #Calculamos la distancia Manhattan dados dos puntos
    return distancia
\end{lstlisting}

Este cambio tambien es importante ya que notemos primero en la version 1 dependemos mucho de la posicion actual del agente(\textbf{nodo_actual}) y la posición de la salida(\textbf{nodo_obejtivo}) como parametros en cambio en la version 2, al eliminar esto hacemos uso de dos funciones \textbf{punto_de_partida} con esta obtenemos las coordenadas de la posición inical del agente y \textbf{salida_laberinto} con esta obtenemos las coordenadas de la salida del laberinto (si la hay), y notemos que al hacer este cambio hacemos que la funcion de distancia Manhattan sea mas versatil, es decir, podemos usarla en diferentes laberintos, pasando como parametro algún laberinto, y asi no modificar la funcion, con esto tambien añadimos que mejora la modularidad del codigo, haciendo que el codigo pueda ser usado en un futuro, y sea mas facil de comprender.

Las siguiente funciones son metodos auxiliares que nos ayudan a encontrar la posición inicial del agente y la salida del laberitno respectivamente.
Se realizo este añadido con la finalidad de que sea más fácil de entender y comprender como estamos buscando las coordenadas de la entrada y salida del laberinto, además de hacer que nuestro código sea mas flexible

\begin{lstlisting}[language=Python, caption=Función auxiliar: punto_de_partida]
    def punto_de_partida(laberinto):
    """
    Encuentra la posición inical del agente en el laberinto

    Recibe:
        laberinto (list): El laberinto representado como una matriz.

    Devuelve:
        tupla: La posición(coordenadas) inicial del agente
    """
    try:
        for i in range(len(laberinto)):
            for j in range(len(laberinto[0])):
                if laberinto[i][j] == "E":  # Comprobamos si encontramos la entrada
                    return (i, j)  # Devolvemos las coordenadas 
    except ValueError:
        print("No se encontró la salida del laberinto.")
\end{lstlisting}

\begin{lstlisting}[language=Python, caption=Función auxiliar: salida_laberinto]
    def salida_laberinto(laberinto):
    """
    Encuentra la posición de la salida del laberinto

    Recibe:
        laberinto (list): El laberinto representado como una matriz.

    Devuelve:
        tupla: La posición(coordenadas) de la salida del laberinto.
    """
    try:
        for i in range(len(laberinto)):
            for j in range(len(laberinto[0])):
                if laberinto[i][j] == "S":  # Comprobamos si encontramos la salida
                    return (i, j)  # Devolvemos las coordenadas 
    except ValueError:
        print("No se encontró la salida del laberinto.")
\end{lstlisting}

Otros pequeños cambios que se hicieron fue que en lugar se usar una condición para ejecutar el algoritmo, eso se hizo simpelemnete por comodidad y que sea mas claro.

\begin{lstlisting}[language=Python, caption=Versión 1: ejecución algoritmo]
	if camino:
    	print("Camino encontrado:", camino)
	else:
    	print("No se encontró un camino válido.")
\end{lstlisting}

\begin{lstlisting}[language=Python, caption=Versión 2: ejecución algoritmo]
	agente = Agente([0, 0])
	buscar_camino_A_estrella(agente, laberinto)
\end{lstlisting}

