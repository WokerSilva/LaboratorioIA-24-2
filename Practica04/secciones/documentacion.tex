\section{Documentación}

\subsection*{Pseudocódigo $A^{*}$}

\textbf{Es importante mencionar que el algoritmo que la ayudante Tania presentó en su clase grabada fue el que utilizamos 
para implementar el algoritmo.}\\ 

\begin{itemize}
    \item[] \textbf{Entrada:} Posición inicial, posición objetivo
    \item[] \textbf{Salida:} Camino desde la posición inicial hasta la posición objetivo
    \item[] \textbf{Variables:}
    \begin{itemize}
        \item[] \texttt{lista\_cerrada}: Lista de nodos ya examinados
        \item[] \texttt{lista\_abierta}: Lista de nodos por examinar
        \item[] \texttt{nodo\_actual}: Nodo actual que se está evaluando
        \item[] \texttt{vecino}: Cada uno de los vecinos del nodo actual
    \end{itemize}
\end{itemize}

\textbf{Algoritmo:}

\begin{enumerate}
    \item \textbf{Inicializar:}
        \begin{itemize}
            \item \texttt{lista\_cerrada} vacía
            \item Agregar nodo inicial a \texttt{lista\_abierta}
        \end{itemize}
        
    \item \textbf{Mientras} \texttt{lista\_abierta} no esté vacía:
        \begin{enumerate}
            \item \textbf{Obtener} \textit{nodo\_actual} con el menor valor de $f$ de \texttt{lista\_abierta}
            \item \textbf{Eliminar} \texttt{nodo\_actual} de \texttt{lista\_abierta}
            \item \textbf{Agregar} \texttt{nodo\_actual} a \texttt{lista\_cerrada}
            \item \textbf{Obtener vecinos} de \texttt{nodo\_actual}
            \item \textbf{Para cada vecino:}
                \begin{itemize}
                    \item \textbf{Si} vecino es la posición objetivo:
                    
                        \begin{itemize}
                            \item \textbf{Retornar camino} desde \texttt{nodo\_actual}
                        \end{itemize}
                    \item \textbf{Calcular} $g$ del vecino
                    \item \textbf{Calcular} $h$ del vecino
                    \item \textbf{Calcular} $f$ del vecino
                    \item \textbf{Si} vecino ya está en \texttt{lista\_abierta} y $f$ del vecino en \texttt{lista\_abierta} es mayor:
                        \begin{itemize}
                            \item \textbf{Actualizar} $f$ del vecino en \texttt{lista\_abierta}
                            \item \textbf{Actualizar} padre del vecino en \texttt{lista\_abierta}
                        \end{itemize}
                    \item \textbf{Si} vecino ya está en \texttt{lista\_cerrada} y $f$ del vecino en \texttt{lista\_cerrada} es mayor:
                        \begin{itemize}
                            \item \textbf{Eliminar} vecino de \texttt{lista\_cerrada}
                            \item \textbf{Agregar} vecino a \texttt{lista\_abierta}
                            \item \textbf{Actualizar} padre del vecino
                        \end{itemize}
                    \item \textbf{Si} vecino no está en ninguna lista:
                        \begin{itemize}
                            \item \textbf{Agregar} vecino a \texttt{lista\_abierta}
                            \item \textbf{Actualizar} padre del vecino
                        \end{itemize}
                \end{itemize}
        \end{enumerate}

\end{enumerate}

\textbf{Fin del algoritmo}
