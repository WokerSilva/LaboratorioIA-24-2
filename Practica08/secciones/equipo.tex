\subsection{Análisis equipo}

\begin{itemize}
    \item Marco Silva Huerta 
    
    Para el modelo de Logistic Regression su matriz de confusión muestra un rendimiento bastante bueno con un bajo número de 
    falsos positivos y falsos negativos, eso le da un equilibrio del $91\%$ en la clasificación de ham y spam. Pero para el 
    modelo SVM la precisión en números es del $0.9343$, esto pasa porque SVM tiene menos falsos positivos que Logistic Regression.\\ 

    Ahora cuando volteamos a ver a Decision Tree Classifier, muestra un desempeño aceptable en relación a los dos primeros 
    pues cuenta  con un $87\%$ de exactitud general, esto se pude explicar ya que Los árboles de decisión son propensos al 
    sobreajuste cuando se entrenan con conjuntos de datos complejos o desbalanceados ya que sus nodos al dividirse pueden 
    tener dificultades para  generalizar patrones en datos que no han visto durante el entrenamiento.\\ 

    Finalmente el modelo Random Forest Classifier muestra un rendimiento muy muy bueno, fue el que mayor positivos verdaderos tuvo 
    con 1163 y pese a tener también un $93\%$ en la exactitud del modelo, queda segundo lugar pues tiene un $0.9318$, ligeramente 
    por debajo de SVM pero arriba de su individual árbol de decisión, esto porque Random Forest utiliza múltiples árboles de decisión 
    en lugar de uno solo, la combinación de las predicciones reduce el riesgo de sobreajuste.\\ 

\end{itemize}