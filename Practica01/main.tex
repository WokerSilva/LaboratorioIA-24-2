\documentclass[a4paper,12pt]{article} 
\usepackage[utf8]{inputenc} % Acentos válidos sin problemas
\usepackage[spanish]{babel} % Idioma

\usepackage[backend=biber, style=alphabetic, sorting=ynt]{biblatex}
\bibliography{bibliografia.bib}
\nocite{*} % Añade todas las referencias de bib sin cita

\input{packet}
\pagestyle{fancy}

\begin{document}%----------------------INICIO DOCUMENTO------------|
%------------------------------------------------------------------|
\begin{titlepage}
\center 
\newcommand{\HRule}{\rule{\linewidth}{0.5mm}} 

%---------------------
%	ESCUDO
%---------------------
\includegraphics[width=4.5cm]{IMA/cienciasWhite.png}

%----------------------------
%	TITULO
%----------------------------
\quad \\[0.2cm]
\textsc{\huge Facultad de Ciencias}\\[.6cm] 
\textsc{\huge Inteligencia Artificial}\\[0.5cm]

%-------------
%	TRABAJO
%-------------
\makeatletter
    \HRule \\ [0.4cm]
        { \huge \bfseries Exploradores de laberinto}\\
    \HRule \\ [0.4cm]
    
\vspace{2mm}

%----------------------------
%	Nombre del Equipo
%----------------------------
\begin{flushleft}
    \Large{Equipo:} \texttt{\Large Skynet Scribes}
\end{flushleft}
%----------------------------
%	Número de practica
%----------------------------
\begin{flushleft}
    \Large{Número de practica:} \texttt{\Large 02}\\[0.8cm]
\end{flushleft}


%-------------------
%	AUTORES
%-------------------
\begin{minipage}{0.8\textwidth}
    \begin{flushright}
        \textbf{\large{Carlos Daniel Cortés Jiménez}}\\    
        420004846        
    \end{flushright}
\end{minipage}

\vspace{5mm}

\begin{minipage}{0.4\textwidth}
        \textbf{\large{Sarah Sophía Olivares García}}\\
        318360638
\end{minipage}
\begin{minipage}{0.4\textwidth}
    \begin{flushright}
        \textbf{\large{Marco Silva Huerta}}\\
        316205326        
    \end{flushright}
\end{minipage}

\vspace{5mm}

\begin{minipage}{0.4\textwidth}
        \textbf{\large{Juan Daniel Barrera Holan}}\\    
        417079372
\end{minipage}
\begin{minipage}{0.4\textwidth}
    \begin{flushright}
        \textbf{\large{Laura Itzel Tinoco Miguel}}\\
        316020189
    \end{flushright}
\end{minipage}

\vspace{10mm}
%-------------------
%	PROFESORES
%-------------------

\begin{minipage}{0.8\textwidth}
    \begin{flushleft} \large
        Profesora: Cecilia Reyes Peña\\
        Ayudante teoría: Karem Ramos Calpulalpan \\
        Ayudante laboratorio: Tania Michelle Rubí Rojas\\                    
    \end{flushleft}
\end{minipage}

\vspace{20mm}

\begin{minipage}{0.4\textwidth}
    %---------------
    %	S E M E S T R E
    %---------------
    \textcolor{white}{Semestre}\\
    \large\textbf{Semestre 2024-2}      
\end{minipage}
\begin{minipage}{0.4\textwidth}
    %---------------
    %	F E C H A
    %---------------
    \begin{flushright}
        {\large Fecha de entrega:\\
         \textbf{28 de Febrero del 2024}}
    \end{flushright}
\end{minipage}

\makeatother

\vfill 
\end{titlepage}

\newpage
\section*{Documentación}
\subsection*{Como funciona el ChatBot}

Antes de iniciar a conversar con Chat Dog necesitamos esto en nuestra computadora:
\begin{itemize}
    \item Versión de python para evitar conflictos con el entorno.
\begin{lstlisting}[language=bash]
    $ python --version
    $ Python 3.7.9    
\end{lstlisting}

\item Navegar hasta la carpeta donde esta el archivo bot.py    
\begin{lstlisting}[language=bash]
    $ cd Practica01/src
    /src $ ls 
    bot.py  conver.json &\textcolor{water}{templates}&  &\textcolor{water}{nenv}&
\end{lstlisting}

\item Vamos a instalar flask para que nuestro bot se vea correctamente en la web
\begin{lstlisting}[language=bash]    
    /src $ pip install flask
\end{lstlisting}
\begin{itemize}
    \item (Se descargaran los archivos necesarios)
\end{itemize}

\item Lo que sigue es activar nuestro entorno 
\begin{itemize}
    \item Para Windows
\begin{lstlisting}[language=bash]    
    \src $ venv\Scripts\activate
    &\textcolor{ForestGreen}{(venv)}& \src $
\end{lstlisting}
\item Para Linux
\begin{lstlisting}[language=bash]    
    \src $ source_venv/bin/activate
    &\textcolor{ForestGreen}{(venv)}& \src $
\end{lstlisting}

\item Sabremos que esta activo el entorno al ver del lado izquierdo: (venv)

\end{itemize}

\item El siguiente paso es ejecutar el bot con el comando: \texttt{python bot.py}
\begin{lstlisting}[language=bash]    
    &\textcolor{ForestGreen}{(venv)}& \src $ ls
    bot.py  conver.json &\textcolor{water}{templates}&  &\textcolor{water}{nenv}&
    &\textcolor{ForestGreen}{(venv)}& \src $ python bot.py
\end{lstlisting}
\begin{itemize}
    \item A continuación veremos varias lineas como estas
\begin{lstlisting}[language=bash]    
    &\textcolor{ForestGreen}{(venv)}& \src $ python bot.py
    [nltk_data] Downloading package averaged_perceptron_tagger to
    [nltk_data]     C:\Users\youruser\AppData\Roaming\nltk_data...
    List Trainer: [####################] 100%
    * Serving Flask app 'bot'
    * Debug mode: on    
\end{lstlisting}
    \item La siguiente linea es importante porque es el puerto donde verás a Chat-Dog
\begin{lstlisting}[language=bash]        
    * Serving Flask app 'bot'
    * Debug mode: on    
    * Running on http://127.0.0.1:5000
\end{lstlisting}
    \item Tal vez en tu computadora sea otro puerto, pero así debe verse lo que copies y peques en tu navegador de preferencia (\texttt{http://127.0.0.1:5000})
\begin{lstlisting}[language=bash]        
    * Running on http://127.0.0.1:5000
    [nltk_data]   Package stopwords is already up-to-date!
    List Trainer: [####################] 100%
    * Debugger is active!


\end{lstlisting}
\end{itemize}

\item Esto te deberá permitir ver la interfaz de Chat-Dog
\item Y listo podrás empezar a conversar con él 
\item Para terminal el proceso basta con que en la terminal ejecutes: \texttt{Ctrl+C}
\begin{lstlisting}[language=bash]        
    * Running on http://127.0.0.1:5000
    [nltk_data]   Package stopwords is already up-to-date!
    List Trainer: [####################] 100%
    * Debugger is active!

    Ctrl+C
    &\textcolor{ForestGreen}{(venv)}& \src $
\end{lstlisting}

\end{itemize}



\section*{Investigación}

\section*{Propósito}
El propósito principal del chatbot es brindar un servicio educativo e informativo sobre animales domésticos y sus cuidados veterinarios. Busca ser una herramienta que ayude a los usuarios a comprender mejor cómo cuidar adecuadamente a sus mascotas.

\textbf{Funcionalidades básicas:}
\begin{itemize}
    \item \textbf{Mensaje de saludo:} El chatbot debe ser capaz de saludar a los usuarios de manera amigable, utilizando una variedad de saludos para crear una experiencia acogedora desde el inicio de la interacción.
    
    \item \textbf{Descripción de servicios:} Proporciona una descripción clara y concisa de los servicios que ofrece 

    \item \textbf{Datos interesantes:} Proporciona curiosidades y datos interesantes sobre animales domésticos, como sus habilidades naturales, comportamientos únicos,etc.

    \item \textbf{Despedida cordial:} Al finalizar la interacción manda un mensaje de despedida cordial, agradeciéndoles por utilizar el servicio.
\end{itemize}\section*{Implementación}



% ----------------------|
% Referencias           |
% pdflatex main.tex     |
% biber main            |
% pdflatex main.tex     |
\newpage %              |
\thispagestyle{fancyref}
\printbibliography %    |
% ----------------------|

\end{document}%----------------------F I N DOCUMENTO---------------|
%------------------------------------------------------------------|
