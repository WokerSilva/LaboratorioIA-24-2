\documentclass[a4paper,12pt]{article} 
\usepackage[utf8]{inputenc} % Acentos válidos sin problemas
\usepackage[spanish]{babel} % Idioma

\usepackage[backend=biber, style=alphabetic, sorting=ynt]{biblatex}
\bibliography{bibliografia.bib}
\nocite{*} % Añade todas las referencias de bib sin cita

\input{packet}
\pagestyle{fancy}

\begin{document}%----------------------INICIO DOCUMENTO------------|
%------------------------------------------------------------------|
\begin{titlepage}
\center 
\newcommand{\HRule}{\rule{\linewidth}{0.5mm}} 

%---------------------
%	ESCUDO
%---------------------
\includegraphics[width=4.5cm]{IMA/cienciasWhite.png}

%----------------------------
%	TITULO
%----------------------------
\quad \\[0.2cm]
\textsc{\huge Facultad de Ciencias}\\[.6cm] 
\textsc{\huge Inteligencia Artificial}\\[0.5cm]

%-------------
%	TRABAJO
%-------------
\makeatletter
    \HRule \\ [0.4cm]
        { \huge \bfseries Exploradores de laberinto}\\
    \HRule \\ [0.4cm]
    
\vspace{2mm}

%----------------------------
%	Nombre del Equipo
%----------------------------
\begin{flushleft}
    \Large{Equipo:} \texttt{\Large Skynet Scribes}
\end{flushleft}
%----------------------------
%	Número de practica
%----------------------------
\begin{flushleft}
    \Large{Número de practica:} \texttt{\Large 02}\\[0.8cm]
\end{flushleft}


%-------------------
%	AUTORES
%-------------------
\begin{minipage}{0.8\textwidth}
    \begin{flushright}
        \textbf{\large{Carlos Daniel Cortés Jiménez}}\\    
        420004846        
    \end{flushright}
\end{minipage}

\vspace{5mm}

\begin{minipage}{0.4\textwidth}
        \textbf{\large{Sarah Sophía Olivares García}}\\
        318360638
\end{minipage}
\begin{minipage}{0.4\textwidth}
    \begin{flushright}
        \textbf{\large{Marco Silva Huerta}}\\
        316205326        
    \end{flushright}
\end{minipage}

\vspace{5mm}

\begin{minipage}{0.4\textwidth}
        \textbf{\large{Juan Daniel Barrera Holan}}\\    
        417079372
\end{minipage}
\begin{minipage}{0.4\textwidth}
    \begin{flushright}
        \textbf{\large{Laura Itzel Tinoco Miguel}}\\
        316020189
    \end{flushright}
\end{minipage}

\vspace{10mm}
%-------------------
%	PROFESORES
%-------------------

\begin{minipage}{0.8\textwidth}
    \begin{flushleft} \large
        Profesora: Cecilia Reyes Peña\\
        Ayudante teoría: Karem Ramos Calpulalpan \\
        Ayudante laboratorio: Tania Michelle Rubí Rojas\\                    
    \end{flushleft}
\end{minipage}

\vspace{20mm}

\begin{minipage}{0.4\textwidth}
    %---------------
    %	S E M E S T R E
    %---------------
    \textcolor{white}{Semestre}\\
    \large\textbf{Semestre 2024-2}      
\end{minipage}
\begin{minipage}{0.4\textwidth}
    %---------------
    %	F E C H A
    %---------------
    \begin{flushright}
        {\large Fecha de entrega:\\
         \textbf{28 de Febrero del 2024}}
    \end{flushright}
\end{minipage}

\makeatother

\vfill 
\end{titlepage}

\newpage
% ------------------------------------------------------------------------------|
% -----------------------------------------------------------------------------|
% ----------------------------------------------------------------------------|
\section{Investigación}
% ----------------------------------------------------------------------------\
% -----------------------------------------------------------------------------\
% ------------------------------------------------------------------------------\

\begin{center}
    {\large ChatterBot}
\end{center}

La biblioteca ChatterBot de Python está diseñada para responder de manera automatizada a las entradas que de él usuario, con la ayuda 
de varios algoritmos de aprendizaje automático que generan una variedad de respuestas, lo que permite que al desarrollar chatbots se 
puedan dar respuestas apropiadas en diferentes contextos. El chatbot también va a poder mejorar su rendimiento conforme aprende con 
el paso del tiempo, presenta independencia de lenguaje, de esta manera se puede entrenar para hablar en cualquier idioma


\subsection{Funcionamiento}

Un chatbot no entrenado de ChatterBot inicia sin conocimiento de cómo debe comunicarse con el usuario,se pueden introducir frases y 
respuestas para que el chatbot aprenda qué debe responder en cada ocasión, cada vez que el usuario ingresa una entrada el chatbot 
guarda las entradas y respuestas con las que va tratando y con esos mismos datos genera respuestas automatizadas relevantes, cuando 
recibe una nueva entrada, compara la nueva entrada con datos anteriores, así chatbot puede seleccionar una respuesta que esté vinculada 
a la entrada conocida más cercana posible.

\subsection{Capacidades}

Como se mencionó anteriormente, algunas de sus capacidades es son el aprendizaje automático así como la independencia 
de lenguaje. Entre otras de sus capacidades están:

\begin{itemize}
    \item \textbf{Almacenamiento y consulta de datos}
    
    ChatterBot cuenta con con módulos llamados storage adapters los cuales permiten interactuar con el sistema de almacenamiento 
    y proporcionan una abstracción, en su version 1.0.8 tiene los siguientes storage adapters; mongodb, sql$\_$storage, django$\_$storage. 
    De esta manera permite mantener un registro de las interacciones y facilita la gestión de datos.

    \item \textbf{Procesamiento de lenguaje natural(NLP siglas en ingles)}
    
    ChatterBot utiliza técnicas de procesamiento de lenguaje natural para poder entender y responder a los mensajes de los usuarios, 
    de esta manera analiza las preguntas y comentarios de los usuarios para generar respuestas relevantes y coherentes.

    \item \textbf{Personalización}
    
    Permite al desarrollador personalizar las respuestas y entrenar al chatbot con datos específicos para que 
    responda de manera mas precisa y personalizada. 
\end{itemize}


\subsection{Limitaciones}

\textbf{Falta de inteligencia emocional:} Los chatbot presentan dificultad para comprender las emociones humanas, 
sarcasmo… Lo cual puede representar un problema a la hora de interpretar el objetivo de las preguntas.\\

\textbf{Dependencia de los datos de entrenamiento:} La calidad y variedad de los datos de entrenamiento de ChatterBot 
influyen en gran medida con la calidad de las respuestas que genere. Los datos sesgados o incompletos pueden dar 
lugar a respuestas incorrectas o sesgos involuntarios en el comportamiento del chatbot. Por lo cual depende mucho de 
un entrenamiento adecuado para que brinde respuestas mas precisas.\\

\textbf{Incapacidad para afrontar escenarios inesperados:}  Al igual que otros chatbots, ChatterBot funciona 
siguiendo reglas y patrones predefinidos, cuando se enfrenta a situaciones que no se espera, lo mas probable es que 
no sepa como responder de manera adecuada, de respuestas fuera de contexto e incluso que responda con información 
poco precisa, por lo cual la base de datos del ChatterBot requiere actualizarse de manera frecuente.

\textbf{Limitaciones en el contexto:} Dado que ChatterBot en su versión 1.0.8, solo trabaja con el contexto de 
la entrada anterior esto puede ser un problema para seguir el hilo de una conversación por ejemplo. 




% ------------------------------------------------------------------------------|
% -----------------------------------------------------------------------------|
% ----------------------------------------------------------------------------|
\section{Propósito}
% ----------------------------------------------------------------------------\
% -----------------------------------------------------------------------------\
% ------------------------------------------------------------------------------\

El propósito principal del chatbot es brindar un servicio educativo e informativo sobre animales domésticos, principalmente de perros. Busca ser una herramienta que ayude a los usuarios a comprender mejor cómo cuidar adecuadamente a sus mascotas.\\

\textbf{Funcionalidades básicas:}
\begin{itemize}
    \item \textbf{Mensaje de saludo:} El chatbot debe ser capaz de saludar a los usuarios de manera amigable, utilizando una variedad de saludos para crear una experiencia acogedora desde el inicio de la interacción.
    
    \item \textbf{Descripción de servicios:} Proporciona una descripción clara y concisa de los servicios que ofrece 

    \item \textbf{Datos interesantes:} Proporciona curiosidades y datos interesantes sobre animales domésticos, como sus habilidades naturales, comportamientos únicos,etc.

    \item \textbf{Despedida cordial:} Al finalizar la interacción manda un mensaje de despedida cordial, agradeciéndoles por utilizar el servicio.
\end{itemize}


% ------------------------------------------------------------------------------|
% -----------------------------------------------------------------------------|
% ----------------------------------------------------------------------------|
\section{Documentación}
% ----------------------------------------------------------------------------\
% -----------------------------------------------------------------------------\
% ------------------------------------------------------------------------------\

\subsection{Organización del proyecto}


Dentro del archivo \texttt{.zip}, el proyecto se compone de la siguiente forma\\

\begin{minipage}[t]{6.8cm}
\begin{itemize}
    \item Carpeta Principal: PRACTICA01
    \begin{itemize}
        \item Carpeta: IMA
        \item Carpeta: src
        \begin{itemize}
            \item Carpeta: venv
            \item Carpeta: templates            
            \begin{itemize}
                \item HTML: index.html
            \end{itemize}
            \item PYTHON: bot.py
            \item JSON: conver.json
            \item BD : db.sqlite3
        \end{itemize}
        \item PDF : practica01.pdf
    \end{itemize}
\end{itemize}
\end{minipage}
\hspace{2mm}
\begin{minipage}[t]{9cm}
    \begin{itemize}
        \item[] Carpeta Principal: PRACTICA01
        \begin{itemize}
            \item[] Guarda las imágenes del proyecto 
            \item[] Archivos ejecutables para el bot
            \begin{itemize}
                \item[] Contiene la configuración del entorno
                \item[] Contiene el front del chatbot            
                \begin{itemize}
                    \item[] HTML: index.html
                \end{itemize}
                \item[] Lógica del programa
                \item[] Es información del entrenamiento
                \item[] Guarda el \textit{aprendizaje} del bot 
            \end{itemize}
            \item[] PDF : practica01.pdf
        \end{itemize}
    \end{itemize}
\end{minipage}

\subsection{Como ejecutar Chat Dog}

Antes de iniciar a conversar con Chat Dog necesitamos esto en nuestra computadora:
\begin{enumerate}
    \item Iniciamos nuestra terminal
    \item Revisamos la versión de python para evitar conflictos con el entorno.
\begin{lstlisting}[language=bash]
    &\textcolor{rubber}{Skynet}& : $ python --&\textcolor{codecommentTerminal}{version}&
    &\textcolor{rubber}{Skynet}& : $  Python 3.7.9    
\end{lstlisting}

\item Navegar hasta la carpeta donde esta el archivo bot.py    
\begin{lstlisting}[language=bash]
    &\textcolor{rubber}{Skynet}& : $ cd Practica01/src
    &\textcolor{rubber}{Skynet}& : &\textcolor{Meadow}{/src}& $  ls 
    bot.py  conver.json &\textcolor{blueRY}{templates}&  &\textcolor{blueRY}{nenv}&
\end{lstlisting}

\item Vamos a instalar flask para que nuestro bot se vea correctamente en la web
\begin{lstlisting}[language=bash]    
    &\textcolor{rubber}{Skynet}& : &\textcolor{Meadow}{/src}& $ pip install flask
\end{lstlisting}
\begin{itemize}
    \item (Se descargaran los archivos necesarios)
\end{itemize}

\item Lo que sigue es activar nuestro entorno, primero crearemos la carpeta 
\begin{itemize}
    \item Para Windows
\begin{lstlisting}[language=bash]    
    &\textcolor{rubber}{Skynet}& : &\textcolor{Meadow}{/src}& $ python -m venv venv    
\end{lstlisting}
    \item Para Linux
\begin{lstlisting}[language=bash]    
    &\textcolor{rubber}{Skynet}& : &\textcolor{Meadow}{/src}& $ python -m venv venv    
\end{lstlisting}
\end{itemize}

\item Ahora vamos a activar el entorno

\begin{itemize}
    \item Para Windows
\begin{lstlisting}[language=bash]    
    &\textcolor{rubber}{Skynet}& : &\textcolor{Meadow}{/src}& $ venv\Scripts\activate
    &\textcolor{ForestGreen}{(venv)}& &\textcolor{rubber}{Skynet}& : &\textcolor{Meadow}{/src}& $
\end{lstlisting}
\item Para Linux
\begin{lstlisting}[language=bash]    
    &\textcolor{rubber}{Skynet}& : &\textcolor{Meadow}{/src}& $ source_venv/bin/activate
    &\textcolor{ForestGreen}{(venv)}& &\textcolor{rubber}{Skynet}& : &\textcolor{Meadow}{/src}& $
\end{lstlisting}

\item Sabremos que esta activo el entorno al ver del lado izquierdo: (venv)

\end{itemize}

\item Ahora instalaremos CharrterBot

\begin{itemize}
    \item Para Windows
\begin{lstlisting}[language=bash]    
    &\textcolor{ForestGreen}{(venv)}& &\textcolor{rubber}{Skynet}& : &\textcolor{Meadow}{/src}& $ python -m pip install chatterbot==1.0.4 pytz
    &\textcolor{ForestGreen}{(venv)}& &\textcolor{rubber}{Skynet}& : &\textcolor{Meadow}{/src}& $
\end{lstlisting}
\item Para Linux
\begin{lstlisting}[language=bash]    
    &\textcolor{ForestGreen}{(venv)}& &\textcolor{rubber}{Skynet}& : &\textcolor{Meadow}{/src}& $ python -m pip install chatterbot==1.0.4 pytz
    &\textcolor{ForestGreen}{(venv)}& &\textcolor{rubber}{Skynet}& : &\textcolor{Meadow}{/src}& $
\end{lstlisting}
\end{itemize}

\item El siguiente paso es ejecutar el bot con el comando: \texttt{python bot.py}
\begin{lstlisting}[language=bash]    
    &\textcolor{ForestGreen}{(venv)}& &\textcolor{rubber}{Skynet}& : &\textcolor{Meadow}{/src}& $ ls
    bot.py  conver.json &\textcolor{blueRY}{templates}&  &\textcolor{blueRY}{nenv}&
    &\textcolor{ForestGreen}{(venv)}& &\textcolor{rubber}{Skynet}& : &\textcolor{Meadow}{/src}& $ python bot.py
\end{lstlisting}
\begin{itemize}
    \item A continuación veremos varias lineas como estas
\begin{lstlisting}[language=bash]    
    &\textcolor{ForestGreen}{(venv)}& &\textcolor{rubber}{Skynet}& : &\textcolor{Meadow}{/src}& $ python bot.py
    [nltk_data] Downloading package averaged_perceptron_tagger to
    [nltk_data]     C:\Users\youruser\AppData\Roaming\nltk_data...
    List Trainer: [####################] 100%
    * Serving Flask app 'bot'
    * Debug mode: on    
\end{lstlisting}
    \item La siguiente linea es importante porque es el puerto donde verás a Chat-Dog
\begin{lstlisting}[language=bash]            
    * Running on http://127.0.0.1:5000
\end{lstlisting}
    \item Tal vez en tu computadora sea otro puerto, pero así debe vers. Copia y pega en tu navegador de preferencia (\texttt{http://127.0.0.1:5000})
\begin{lstlisting}[language=bash]        
    * Running on http://127.0.0.1:5000
    [nltk_data]   Package stopwords is already up-to-date!
    List Trainer: [####################] 100%
    List Trainer: [####################] 100%
    List Trainer: [####################] 100%
    List Trainer: [####################] 100%
    List Trainer: [####################] 100%
    * Debugger is active!



\end{lstlisting}
\end{itemize}

\item Esto te deberá permitir ver la interfaz de Chat-Dog
\item Y listo podrás empezar a conversar con él 
\item Para terminal el proceso basta con que en la terminal ejecutes: \texttt{Ctrl+C}
\begin{lstlisting}[language=bash]        
    * Running on http://127.0.0.1:5000
    [nltk_data]   Package stopwords is already up-to-date!
    List Trainer: [####################] 100%
    List Trainer: [####################] 100%
    * Debugger is active!

    Ctrl+C
    &\textcolor{ForestGreen}{(venv)}& &\textcolor{rubber}{Skynet}& : &\textcolor{Meadow}{/src}& $
\end{lstlisting}

\end{enumerate}



% ----------------------|
% Referencias           |
% pdflatex main.tex     |
% biber main            |
% pdflatex main.tex     |
\newpage %              |
\thispagestyle{fancyref}
\printbibliography %    |
% ----------------------|

\end{document}%----------------------F I N DOCUMENTO---------------|
%------------------------------------------------------------------|
