%-----------------------------------INICIO DE PACKETES-------------------/
%-----------------------------------------------------------------------/
\usepackage{amsmath}   % Matemáticas: Comandos extras(cajas ecuaciones) |
\usepackage{amsthm}
\usepackage{amssymb}   % Matemáticas: Símbolos matemáticos              |
\usepackage{datetime}  % Agregar fechas                                 |
\usepackage{graphicx}  % Insertar Imágenes                              |
%\usepackage{biblatex} % Bibliografía                                   |
\usepackage{multicol}  % Creación de tablas                             |
\usepackage{longtable} % Tablas más largas                              |
\usepackage{xcolor}    % Permite cambiar colores del texto              |
\usepackage{tcolorbox} % Cajas de color                                 |
\usepackage{setspace}  % Usar espacios                                  |
\usepackage{fancyhdr}  % Para agregar encabezado y pie de página        |
\usepackage{lastpage}  %                                                 |
\usepackage{float}     % Flotantes                                      |
\usepackage{soul}      % "Efectos" en palabras                          |
\usepackage{hyperref}  % Para usar hipervínculos                        |
\usepackage{caption}   % Utilizar las referencias                       |
\usepackage{subcaption} % Poder usar subfiguras                         |
\usepackage{multirow}  % Nos permite modificar tablas                   |
\usepackage{array}     % Permite utilizar los valores para multicolumn  |
\usepackage{booktabs}  % Permite modificar tablas                       |
\usepackage{diagbox}   % Diagonales para las tablas                     |
\usepackage{colortbl}  % Color para tablas                              |
\usepackage{listings}  % Escribir código                               |
\usepackage{mathtools} % SIMBOLO :=                                     |
\usepackage{enumitem}  % Modificar items de Listas                      |
\usepackage{tikz}      %                                                |
\usepackage{lipsum}    % for auto generating text                       |
\usepackage{afterpage} % for "\afterpage"s                              |
\usepackage{pagecolor} % With option pagecolor={somecolor or none}|     |
\usepackage{xpatch}    % Color de lineas C & F
%\usepackage{glossaries} %                                              |
\usepackage{lastpage}    %                                              |   
\usepackage{csquotes}    %                                              |
%-----------------------------------------------------------------------\
%-----------------------------------FIN--- DE PACKETES-------------------\

\usepackage{pgfplots}     %                                             |
\pgfplotsset{compat=1.18} %           
\usepackage{etoolbox}
\usepackage{tikz,times}
\usepackage{verbatim}
\usetikzlibrary{mindmap,trees,backgrounds}
%--------------------------------/
%-------------------------------/
\usepackage[                 %   |
  headheight=15pt,  %            |
  letterpaper,  % Tipo de pag.   |
  left =1.5cm,  %  < 1 >         |
  right =1.5cm, %  < 1 >         | MARGENES DE LA PAGINA
  top =2cm,     %  < 1.5 >       |
  bottom =1.5cm %  < 1.5 >       |
]{geometry}     %                |
%-------------------------------\
%--------------------------------\

%----------------------------------------------------------------------/
%-------------------Encabezado y Pie de Pagina-----------------------/ |
%--------------------------------------------------------------------\ |
%\fancyhf{}
%\pagestyle{fancy}

\fancypagestyle{firstpage}{  
    \fancyhead[L]{}
    \fancyhead[R]{}     
    \fancyfoot[L]{}
    \fancyfoot[C]{}
    \fancyfoot[R]{\thepage\ de \pageref*{LastPage}}    
    \renewcommand{\headrulewidth}{0pt} 
    \xpretocmd\headrule{}{}{\PatchFailed}
}

\fancypagestyle{fancy}{  
    \fancyhead[L]{\textbf{Semestre: 2024-2}}
    \fancyhead[C]{}     
    \fancyhead[R]{}     

    \fancyfoot[L]{\texttt{Skynet Scribes}}
    \fancyfoot[C]{\texttt{IA}}
    \fancyfoot[R]{\thepage\ de \pageref*{LastPage}}

    \renewcommand{\headrulewidth}{1pt} 
    \xpretocmd\headrule{}{}{\PatchFailed}
    \renewcommand{\footrulewidth}{1.5pt} 
    \xpretocmd\footrule{}{}{\PatchFailed}
}

\fancypagestyle{fancyref}{  
    \fancyhead[L]{}
    \fancyhead[R]{}     
    \fancyfoot[L]{\texttt{Skynet Scribes}}
    \fancyfoot[C]{\texttt{IA}}
    \fancyfoot[R]{\thepage\ de \pageref*{LastPage}}    
    \renewcommand{\headrulewidth}{0pt} 
    \xpretocmd\headrule{}{}{\PatchFailed}
}
%--------------------------------------------------------------------\ |
%-------------------Encabezado y Pie de Pagina-----------------------/ |
%------------------------------------------------------------FIN----/


%--------------------------------------------------------------------/
%------------------- LISTA DE COLORES ------------------------------/ 
\definecolor{ProcessBlue}{RGB}{0,176,240}
\definecolor{NavyBlue}{RGB}{0,110,184}
\definecolor{Cyan}{RGB}{0,174,239}
\definecolor{MidnightBlue}{RGB}{0,103,49}
\definecolor{ForestGreen}{RGB}{0,155,85}
\definecolor{Goldenrod}{RGB}{255,223,66}
\definecolor{YellowGreen}{RGB}{152,204,112}
\definecolor{Sepia}{RGB}{103,24,0}
\definecolor{Peach}{RGB}{247,150,90}
\definecolor{CarnationPink}{RGB}{242,130,180}
\definecolor{Fuchsia}{RGB}{140,54,140}
\definecolor{WildStrawberry}{RGB}{238,41,103}

\definecolor{Grass}{RGB}{41,238,53}
\definecolor{Meadow}{RGB}{6,243,67}
\definecolor{jellyfish}{RGB}{109,14,130}
\definecolor{rubber}{RGB}{229,27,232}
\definecolor{bullet}{RGB}{225,31,90}
\definecolor{midnight}{RGB}{31,90,225}
\definecolor{sun}{RGB}{241,152,7}
\definecolor{water}{RGB}{16,229,183}

%------------------- COLORES CÓDIGO ---------------------- |
%------------------- COLORES JAVA ---------------------- |
\definecolor{backcolour}{RGB}{6,6,6} 
%\definecolor{backcolour}{RGB}{181,181,181} 
\definecolor{codeclassjava}{RGB}{246,113,59}
\definecolor{codegreen}{RGB}{17,225,48}
\definecolor{codenumizq}{RGB}{17,17,17}
\definecolor{codestringjava}{RGB}{51,240,234}
\definecolor{codesymboljava}{RGB}{255,5,0} 
\definecolor{yellowpoint}{RGB}{244,235,100} 
%------------------- COLORES JAVA ---------------------- |
%------------------- COLORES PYTHON -------------------- |
\definecolor{backcolourPY}{RGB}{6,6,6} 
%\definecolor{backcolour}{RGB}{181,181,181} 
\definecolor{codegreenPY}{RGB}{17,225,48}
\definecolor{codeclassPY}{RGB}{246,113,59}
\definecolor{codenumizq}{RGB}{17,17,17}
\definecolor{codestringPY}{RGB}{90,128,220}
\definecolor{codesymboljava}{RGB}{255,5,0} 
%------------------- COLORES PYTHON -------------------- |
%------------------- COLORES TERMINAL-------------------- |
\definecolor{backcolourTerminal}{rgb}{0.0, 0.0, 0.0} % Negro
\definecolor{codeclassTerminal}{rgb}{1.0, 1.0, 1.0} % Blanco
\definecolor{codestringTerminal}{rgb}{0.0, 0.6, 0.0} % Verde
\definecolor{codecommentTerminal}{rgb}{0.5, 0.5, 0.5} % Gris
\definecolor{codenumizqTerminal}{rgb}{0.0, 0.0, 1.0} % Azul
\definecolor{codeoptionTerminal}{rgb}{0.4, 0.4, 1.0} % Azul claro para opciones
%------------------- COLORES TERMINAL-------------------- |
%------------------- COLORES CÓDIGO ---------------------- |



%------------------- LISTA DE COLORES -------------------------------\
%---------------------------------------------------------------------\

%-------------- ESTILO de Código PYTHON -----------------------------|
\lstdefinestyle{mystylepython}{
    backgroundcolor=\color{backcolourPY},
    commentstyle=\color{codegreenPY},
    keywordstyle=\color{codeclassPY},
    numberstyle=\tiny\color{codenumizq},
    stringstyle=\color{codestringPY},
    basicstyle=\footnotesize\ttfamily\color{white},
    breakatwhitespace=false,
    breaklines=true,
    captionpos=b,
    keepspaces=true,
    numbers=left,
    numbersep=5pt,
    showspaces=false,
    showstringspaces=false,
    showtabs=false,
    tabsize=2,
    escapechar=\&,
    literate=                
        {;}{{\textcolor{yellowpoint}{;}}}1
        {+}{{\textcolor{yellowpoint}{+}}}1
        {-}{{\textcolor{yellowpoint}{-}}}1
        {\{}{{\textcolor{yellowpoint}{\{}}}1
        {\}}{{\textcolor{yellowpoint}{\}}}}1
        {[}{{\textcolor{yellowpoint}{[}}}1
        {]}{{\textcolor{yellowpoint}{]}}}1
        {=}{{\textcolor{yellowpoint}{=}}}1
        {:}{{\textcolor{yellowpoint}{:}}}1
        {<}{{\textcolor{yellowpoint}{<}}}1
        {>}{{\textcolor{yellowpoint}{>}}}1
}
%-------------- ESTILO de Código PYTHON -----------------------------|
%-------------- ESTILO de Código TERMINAL ---------------------------|
\lstdefinestyle{mystyleTerminal}{
    backgroundcolor=\color{backcolourTerminal},
    commentstyle=\color{codecommentTerminal},
    keywordstyle=\color{codeclassTerminal},
    numberstyle=\tiny\color{codenumizqTerminal},
    stringstyle=\color{codestringTerminal},
    basicstyle=\footnotesize\ttfamily\color{codeclassTerminal},
    breakatwhitespace=false,
    breaklines=true,
    captionpos=b,
    keepspaces=true,
    numbers=left,
    numbersep=5pt,
    showspaces=false,
    showstringspaces=false,
    showtabs=false,
    tabsize=4,    
    escapechar=\&,
    literate = 
            {--}{{\textcolor{codeoptionTerminal}{--}}}2    
}
% Usar el estilo de código
\lstset{style=mystyleTerminal}
%-------------- ESTILO de Código TERMINAL ---------------------------|